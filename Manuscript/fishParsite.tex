\documentclass[12pt]{article}
\usepackage[paper=a4paper, margin=1in]{geometry} 

%Required packages
\usepackage{natbib} 
\usepackage{graphicx}
\usepackage{amsmath}
\usepackage{mathtools}
\usepackage{setspace}
\usepackage[document]{ragged2e}
\usepackage{lineno}

\setcounter{secnumdepth}{-1} 
\raggedright


\begin{document}
\setcounter{page}{1}

\textbf{Title}: What controls the range of hosts a fish parasite infects? \\
\vspace{0.5cm}
\textbf{Authors:} Tad Dallas$^{1,2}$, Andrew Park $^{1}$, and John M. Drake$^{1}$ \\
\vspace{0.5cm}
\textbf{Affiliations}: 
\begin{enumerate}
  \item University of Georgia, Odum School of Ecology, 140 E. Green Street, Athens GA, 30602. 
  \item Corresponding author: \texttt{tdallas@uga.edu}
\end{enumerate}


\linenumbers
\doublespacing


\section{Abstract}



\section{Keywords}





\textbf{Take-home messages} \\
\begin{enumerate}
 \item It is possible to predict parasite niche breadth using either host traits/geographic location, or parasite community similarity (best). 
 \item Predictive accuracy does not vary as a function of host specificity (though I only consider parasites with 20 or more occurrence records, but these could all be on the same host species).
 \item It may be possible to predict parasite spillover from invasive hosts to native host communities, or to predict biotic resistance of a community to invasion. 
\end{enumerate}



\textbf{Needs}\\
\begin{enumerate}
 \item Conceptual diagram?
 \item 
\end{enumerate}

\section{Introduction}
 \paragraph{Host-parasite relationships are complex, intimate interactions with lots of impacts}
 
 
 
 
 \paragraph{Being able to predict parasite occurrence is pretty important, for a number of reasons (species invasions/biotic resistance, spillover to human hosts, etc.}
 
 
 
 \paragraph{Previous work and knowledge gap}
  
  What has been done in fish parasite prediction? Cite some Kennedy, some Poulin, and then talk about what Strona has done with regards to predicting parasite co-occurrence. Point out that predictive power wasn't great, and that co-occurrence is a different question from predicting parasite niche breadth. Discuss the differences in my approach from Strona's approach, and set up the question as both a prediction question, and a ``what's important'' question. Specifically, I trained models on geographic variables and host traits, and on just parasite information. Parasite information performed better, regardless of geographic information exclusion. This suggests that patch quality and geographic location may not matter as much as the existing community of parasites you harbor. In other words, prediction is better when only considering the parasite community of a host, rather than anything about the host. 


 \paragraph{Thesis (what I did, what I found)}
 Here, I trained several predictive models to determine parasite host breadth using data on host traits, parasite community information, geographic location. To do this, I examined a large dataset on interactions between freshwater fish and their parasite communities [@strona2013]. The number and identity of host species that a given parasite could infect may be constrained by space (geographic location), patch quality (host characteristics), or through interactions with competing parasites (parasite community structure). 
 
 
 
 
 
 
 
 
 %It's important to note right up front that the importance of parasite community structure cannot be interpreted as evidence for community interactions, as parasites could infect hosts based on their traits, and the parasite community information could just be serving as a proxy for unmeasured host trait variation. However, predicting parasite occurrence based solely on parasite community information does remove some importance of the patch (host), and is easy to sell, as it may be possible to predict spillover of parasites, or the degree of biotic resistance a community offers to a potential invader, simply by having presence-absence data on parasite communities. 

%What constrains the range of hosts that a parasite can infect? Is there a simple range of host functional traits that can determine the likelihood that a parasite infects a given host species? How well can we predict parasite occurrences given \textit{only} host life history traits? How about using solely information on parasite community structure? 

%Does the importance of different host functional traits or parasite community information differ with parasite type? (supplement)

%Since geographic variables are important, what if we try to predict parasite niche breadth in a specific biogeographic region? (supplement) 
 
 
\section{Methods}

 \paragraph{Data and processing}
 We use an existing global database of fish-parasite associations \citep{strona2013} consisting of over 38000 parasite records spanning a large diversity of parasites (Acanthocephala, Cestoda, Monogenea, Nematoda, Trematoda). In order to allow for cross-validation and accurate prediction, we constrained our ananlyses to parasites with a minimum of 20 host records. In other words, we only examined parasites that had been recorded more than 20 times, but these occurrences could be on fewer than 20 host species. The inclusion of duplicate occurrences was only permitted if the parasite was recorded on a host in a different geographic location, based on latitude and longitude values. Our response variable was parasite occurrence (binary), and was predicted using only host life history traits, and geographic location of host capture. Host trait information was obtained through the FishPest database \citep{strona2012, strona2013}, and FishBase \citep{froese2010}. Host traits descriptions are provided in Table \ref{tab:traits}.
 
 
 \paragraph{Predictor variables}
 Areas of occupancy were calculated as follows: for each species, we plotted all available point records on a global grid of 1x1 degrees  Lat-Lon and then we counted the number of grid cells where the species is known to occur.
 
 
 \paragraph{Model formulation}
 We trained a series of models in order to compare predictive performance of different techniques. Each model was trained on 70\% of the data, and accuracy was determined from the remaining 30\%. We generated background data by randomly sampling host species where parasite $i$ was not recorded. To maintain proportional training data, the number of random samples was selected to be five times greater than the occurrence records. 
  
  
 \paragraph{Models used}
 Discuss null predictions scenario, and then go into other algorithms used (brt, svm, lr, rf)

\texttt{Or do I focus on BRT? BRT is a bit old hat. Perhaps I could report on BRT, but include analyses with SVM, LR, and RF in the supplement?}



 
 
  
\section{Results}



  
  
  


\section{Discussion}
 
 
 
 
 
 
 
\section{Acknowledgements}

\bibliographystyle{plainnat}
\bibliography{carp.bib}


\newpage
\section*{Tables}
  \begin{table}[!h]
  \caption{Description and units of variables used to predict parasite occurrences.}
  \begin{tabular}{cccc}
\hline
  \textbf{Variable} &   \textbf{Units} &   \textbf{Description} &   \textbf{Range} \\ 
\hline
Max length      & cm           & Maximum fish species length  & 1 -- 2000 \\ 
Trophic level   & --           & 1 + mean trophic level of food items   &  2 -- 5\\ 
Age at maturity & years        & Age at sexual maturity  & 0.1 -- 34  \\ 
Life span       & years        & Estimated maximum age & 0 -- 145  \\ 
Growth rate     & years$^{-1}$ & Rate to approach asymptotic length & 0.02 -- 9.87 \\ 
Marine          & --           & Is host found in marine habitat? & binary  \\ 
Freshwater      & --           & Is host found in freshwater habitat? & binary \\ 
Brackish        & --           & Is host found in brackish habitat? & binary \\ 
\hline 
Geographic region   & --      &           & 7 unique regions \\ 
Area of occupancy   & -- & Area of suitable habitat occupied &  \\
Latitude            & degrees &         & 1 -- 148 ???? \\ 
Longitude           & degrees &         & 1 -- 359 ????? \\ 
\hline
  \end{tabular}
  \label{tab:traits}
\end{table}


   
   
   

\newpage
\section{Figures}

\begin{figure}[h!]
 %\includegraphics[width=.5\textwidth]{Figures/}
  \caption{Image plot with brt model results sorted by parasite type.  }
 \label{fig:a}
 \end{figure}


 \begin{figure}[h!]
 %\includegraphics[width=.5\textwidth]{Figures/}
  \caption{BRT RC contributions for each variable, or some kind of plot comparing the results from models trained with host traits/geography, and those trained with parasite community measures? }
 \label{fig:b}
 \end{figure}

 \begin{figure}[h!]
 %\includegraphics[width=.5\textwidth]{Figures/}
  \caption{ }
 \label{fig:c}
 \end{figure}












\end{document}
-